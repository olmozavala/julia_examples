%Summary for the preliminaries
\documentclass[doc]{apa}
\usepackage[usenames,dvipsnames]{color}
\usepackage{graphicx}
\usepackage{algorithmic}
\usepackage{multicol}
\usepackage{multirow}
\usepackage{amsmath}
%\usepackage[margin=0.5in]{geometry}



\begin{document}
\tableofcontents
%This is used to never indent
\setlength\parindent{0pt} 

\section{Eqns}

\subsection{\color{blue}Advection}
The \textbf{Advection or Convection} operator is the following:
\begin{equation}
u \cdot \nabla =\nabla \cdot u  = u_x \frac{\partial}{\partial x} + u_y \frac{\partial}{\partial u_y} + u_z \frac{\partial}{\partial u_z}
\end{equation}


For a conserved quantity described by $f$ it is called the  \textbf{continuity equation} and 
it models the motion of a conserved scalar field as it is advected by a known velocity field. 
\begin{equation}
    \frac{df}{dt} + \nabla \cdot f \vec u  = 0   
\end{equation}

In 2D $f(x,y,t)$ then $\vec u = (u_x, u_y)$ and the \textbf{continuity equation} is: 
\begin{equation}
\begin{split}
    \frac{df}{dt} + \nabla \cdot f \vec u  & = 0  \\
    \text{In 2D:} \\
    \frac{df(x,y,t)}{dt} + u_x \frac{\partial f}{\partial x} + u_y \frac{\partial f}{\partial y}&  = 0  
\end{split}
\end{equation}

\subsection{\color{blue}Diffusion/Heat}
It describes the Brownian motion. 
In this case $u$ represents the density of the diffused material. 
\begin{equation}
    \frac{\partial u}{\partial t} = \nabla \cdot [D \nabla u]
\end{equation}

With a constant diffusion coefficient $D$ given by $\lambda$:
\begin{equation}
    \frac{\partial u}{\partial t} = \lambda \nabla^2 u
\end{equation}

\subsection{\color{blue}Advection-Diffusion}
Combination of diffusion and advection.

\begin{equation}
\frac{\partial f}{\partial t} = \lambda \nabla^2 f - \nabla \cdot f \vec u + R
\end{equation}

$R$ defines the sources or sinks of $f$. 

\subsection{\color{blue}Wave equation}
$u(x_1, x_2, \dots, x_n)$ can be the pressure in a liquid or the particles of a vibrating solid. 
\begin{equation}
    \frac{\partial^2 u}{\partial t^2} = c^2 \nabla^2 u = c^2 \left (
                                        \frac{\partial^2 u}{\partial x^2_1} +
                                        \frac{\partial^2 u}{\partial x^2_2} +
                                        \dots +
                                        \frac{\partial^2 u}{\partial x^2_n} 
                                        \right )
\end{equation}
$c$ is the speed of propagation.
\newpage

\subsection{\color{orange}Burgers' Equation}
It occurss in various areas like fluid mechanics, nonlinear acoustics, gas dynamics, and traffic flow.\\
For a given field $u(x,t)$ and diffusion  coefficient $\nu$, the general \textbf{viscous Burgers' equation}:
\begin{equation}
\frac{\partial u}{\partial t} + u\frac{\partial u}{\partial x} = \nu \frac{\partial^2 u}{\partial x^2}
\end{equation}

\textbf{Inviscid Burgers' equation}. Which is a prototype for conservation equations that can develop discontinuities (shock waves).
\begin{equation}
\frac{\partial u}{\partial t} + u\frac{\partial u}{\partial x} = 0
\end{equation}


%---------------------------------------------------------------------------------------------------
\subsection{Boundary Conditions}

\subsubsection{Dirichlet}
Specifies the values of the function along the boundary of the domain.

For ODEs Dirichlet boundary conditions on the interval $[a,b]$ are:
\begin{equation}
y(a) = \alpha, y(b) = \beta
\label{dr:bc}
\end{equation}

For PDEs  Dirichlet boundary conditions on a domain $ \Omega \subset R^n$ are

\begin{equation}
y(x) = f(x) \;\;\; \forall x \in \partial \Omega
\label{dr:bc1}
\end{equation}

\subsubsection{Neumann}
Specifies the values in which the derivative of a solution along the boundary of the domain.

For ODEs Neumann boundary conditions on the interval $[a,b]$ are:
\begin{equation}
y'(a) = \alpha, y'(b) = \beta
\label{neu:bc}
\end{equation}

For PDEs  Neumann boundary conditions on a domain $ \Omega \subset R^n$ are
\begin{equation}
\frac{\partial y}{\partial \mathbf{n}}(x) = f(x) \;\;\; \forall x \in \partial \Omega
\label{newu:bc1}
\end{equation}
where $\mathbf{n}$ denots the normal to the boundary 

\subsubsection{Robin}
Weighted combination of Dirichlet and Neumann


%---------------------------------------------------------------------------------------------------
\section{Consistency/Stability/Convergent/CFL}

\subsection{Consistency}

When $\Delta t, \Delta x \rightarrow 0$ then the error of the numerical method will go to 0. \\

\subsubsection{Stability} 

\begin{itemize}
    \item The solution of an ODE/PDE is stable if a small perturbation does not cause divergence from the solution. 
    \item A numerical method is stable if a small perturbation does not cause the numercial
    solution to diverge without bound.
\end{itemize}

A FD scheme is stable in a region if there is an integer $J$ such that for any positive time $T$
there is a constant $C_T$ such that:
\begin{equation}
||v^n||_{\Delta x} \leq C_T \sum_{j=0}^J ||v^j||_{\Delta x}
\end{equation}
\textbf{Convergent scheme} ?

\textbf{CFL} It is a necessary condition for convergence, if we reduce the $\Delta x$ we need to reduce $\Delta t$.
{\color{red} Important} in the following equations $u_{x_i}$ corresponds
to the speed of the wave. In the advection equation it is
easy, it is v in $ \frac{ \partial u}{ \partial t} + v\frac{\partial u}{\partial x} = 0 $ 

\begin{equation}
C = \Delta t \left (\sum _{i=1}^n \frac{u_{x_i}}{\Delta x_i} \right ) \leq C_{max}
\end{equation}
For explicit solver $C_{max} = 1$, for implicit $C_{max}$ may be larger.  For $C_{max} = 1$
\begin{equation}
\Delta t \leq \frac{1}{\sum _{i=1}^n \frac{u_{x_i}}{\Delta x_i} }
\end{equation}

%---------------------------------------------------------------------------------------------------
\subsection{Methods}
\subsubsection{FTCS}

Replace any time derivative $u_t$ and space derivatives $u_x$ with:
\begin{equation}
u_t \approx \frac{\phi_m^{n+1} - \phi_m^{n}}{\Delta t}, u_x \approx \frac{\phi_{m+1}^n - \phi_{m-1}^n}{2 \Delta x}, 
\end{equation}

\subsubsection{Leapfrog (CTCS) }
Replace any time derivative $u_t$ and space derivatives $u_x$ with:
\begin{equation}
u_t \approx \frac{\phi_m^{n+1} - \phi_m^{n-1}}{2 \Delta t}, u_x \approx \frac{\phi_{m+1}^n - \phi_{m-1}^n}{2 \Delta x}, 
\end{equation}

\subsubsection{Lax-Friedich}
Replace any time derivative $u_t$ and space derivatives $u_x$ with:
\begin{equation}
u_t \approx \frac{\phi_m^{n+1} - \phi_m^{n-1}}{2 \Delta t}, u_x \approx \frac{\phi_{m+1}^n - \phi_{m-1}^n}{2 \Delta x}, 
\end{equation}


\end{document}
